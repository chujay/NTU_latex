%!TEX root = ../thesis.tex
\section{模型假設與符號表示}
\label{c:ch4.2}
根據上一小節所敘述的問題,本小節主要介紹染整製程參數優化模型當中,所需要的假設以及所需要使用的參數及變數符號。
\begin{enumerate}[(1)]
	\item 研究假設:
	\begin{enumerate}[i. ]
		\item 成本模型中,估計電的成本係數以及水的成本係數為定值,不依時間季節而變動。
		\item 由於實驗用的染缸材質,以至於熱散失所消耗的成本過低,所以熱散失成本並不列入考慮。
		\item 假設人為操作上並無差異。
		\item 由於本研究主要著重於影響較大的製程參數因子,故其他的環境參數因子假設為定值。
	\end{enumerate}
	\item 符號定義:
	\begin{enumerate}[]
		\item $EC$:電銷號總成本
		\item $WC$:水消耗總成本
		\item $TC$:機器佔用總時間成本
		\item $C_1$:每消耗一焦耳的電所換算的電消耗成本(台幣)
		\item $C_2$:每消耗一公斤的水所換算的水消耗成本(台幣)
		\item $C_3$:每佔用一分鐘的機台所換算的機器佔用成本(台幣)
		\item $E_i$:為主染程中第$i$段所需消耗的熱能(焦耳),$i=1\ldots3$
		\item $v$:從室溫到第一個目標溫度以及第一個目標溫度到第三個目標溫度的升溫速率($^\circ C/min$)
		\item $S$:為染液的熱容量
		\item $M$:為布的重量(公斤)
		\item $T_L$:室內溫度($^\circ C$)
		\item $T_H$:主染程中最高溫度($^\circ C$)
		\item $t_i$:為主染程中第$i$段機器所需使用的時間(分鐘),$i=1\ldots3$
		\item $c$:處理每公斤的廢水所需要的成本(台幣)
		\item $K$:染色過程所需要的換水次數
		\item $x_A$:在主染程中,從室溫開始升溫需要到達的第一個目標溫度($^\circ C$)
		\item $x_B$:在主染程中,第一個目標溫度以及第二個目標溫度之間的升溫速率($^\circ C/min$)
		\item $x_C$:在主染程中,從第一個目標溫度升溫需要到達的第二個目標溫度($^\circ C$)
		\item $x_D$:在主染程中,當溫度達到最高時,布料浸泡在染液中的時間(分鐘)
		\item $x_E$:在主染程中,每一公斤的布料所需要多少水量調和的浴比值
	\end{enumerate}
\end{enumerate}
