%!TEX root = ../thesis.tex
\chapter{文獻探討}
\label{c:literat}
布料的染整加工程序被認定為一個高耗能型產業,由於從布料製作到染布進入裁切的過程當中會運用到大量的化學藥劑、水、升溫燃料及運作所需的電能,故在本研究當中,我們會藉由調整製程參數來降低能源及資源的使用量,進而減少在染布製成當中的總消耗成本。另外,我們也藉由調整製程參數來提高布料品質,而在品質與降低成本當中取得一個適當的平衡點為本研究的主要目的。本研究計畫的文獻回顧分為「染色製程參數與對色關聯性」及「染色製程參數優化技術」兩個部分,探討目前業界目前常用的參數組合以及,染色製成優化技術的趨勢。