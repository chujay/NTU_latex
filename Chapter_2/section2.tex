%!TEX root = ../thesis.tex
\section{染色製程參數優化技術}
\label{c:ch2.2}
在實驗設計及其他文獻當中,可以知道這些大量的製程參數中哪一些是屬於影響較大,或是哪些參數彼此間具有相互影響的能力。像是\cite{Guo.etc}表示藉由色力度值來做為計算依據下,可以知道助劑的影響對於染色的色力度是呈正比的關係,也就是助劑的增加可增強其固色能力。而由\cite{Wang.etc}可以知道由實驗可得知酸鹼度、溫度及浴比值這些因子的影響會對染色效果有顯著的差異。\shortcite{Ibrahim.etc}也顯示出在特定的酸鹼濃度下,改變其他的製程參數也會有不同的結果。\cite{Wu.etc}將紡織廠在染色過程中所產生的廢棄物,藉由建置成本模型及所希望實行的方法,用基因演算法找出改善成本及成本模型中的最佳平衡點,也就是穩健的製程變數結果,同時也提出減少多餘資源上的花費,可以有效的降低成本。

在本實驗染色製程參數優化技術中,主要的目的是為了改善製程的總成本,因此除了能從實驗結果知道重點參數影響成本的程度外,還會考量實際紡織廠內所花費資源的成本比例,以及其他文獻內所提過的模型。在\cite{Palamutcu}表示紡織產業是非常消耗能源及環境資源的一種產業,故在降低能源上的成本,相對上環境的成本也能下降,故能源的消耗為紡織產業重要的議題之一,而且UNIDO (United Nations Industrial Development Organization)在1992年也表示,一家紡織工廠的電能消耗率染色處理及洗淨為43\%、製作線材的耗電率為93\%,而紡織布料的消耗率為85\%。本研究針對布料染色製程中的參數及成本的優化,考慮電能消耗、其他熱燃料能源消耗、所需耗水量及升溫所需要的能源,配合穩健的製程參數建構模型。