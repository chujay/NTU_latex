%!TEX root = ../thesis.tex
\section{次章節標題}
\label{c:ch2.1}
染整製程的環境下,對於染布的均勻度及製程所造成的成本都會和許多的參數息息相關。但在建置成研究模型時,如果使用大量的參數可以找出較貼近於實際的模型,而且可以藉由此模型找出最好的解。
並且在染整過程中,有許多染色條件會影響到紡織物染後結果,譬如染料的組成、配方組合及染程的參數設計等。其中染料的組成會影響染料的上色速度,配方組合會影響助劑的用量等。而對於染色製程參數對染色結果的影響,染場技術人員仍停留於先前經驗法則所制定出的製程參數水平值,然後再以誤差法通過多次試打樣去嘗試找出符合品質製程之最佳值,不僅浪費太多的時間和成本在經驗學習上,這樣的製程參數穩健性也不高,很容易受現場製程變異的影響。
目前相關文獻,大多只探討顏色染料的組成對染色失敗率或成本的影響,以及配方的改變對紡織布料染色後的品質影響,反而對製程參數的最佳化研究卻鮮有著墨。例如\cite{Harada.etc}發現染浴pH值可以用來設定不同種類的反應染料之吸盡率與固著率。\cite{Guo.etc}在通過對棉紗染色的研究,發現透過助劑的前處理過程可增加其著色力度。\cite{Chen}探討了三種不同類型的常壓可燃型聚酯織維物,在三種不同的分散性染料之染色結果。\cite{Huang.etc}發現當使用不同比例的單體混合接枝成棉織物,施染後棉織物具有不同的色力度值。\shortcite{Jasper.etc}利用類神經網路從染料的光譜吸光度來預測染料的濃度。可以看出上述文獻所探討的幾乎都是染料的組成和配方組合對染色結果之影響。

在過去文獻提到以實驗設計選擇染整製程參數,當中大部分是以紡織廠的過去經驗或是其他文獻做為依據,利用某些製程參數以原先的模型參數為基準,再進行微調的方式來做改善。像是\cite{Harada.etc}表示酸鹼值及溫度等面向,對於吸收率及固色率具有一定程度上的影響。\cite{jy2007dyeing}表示在尼龍布料的染整製程中,藉由微調酸鹼值、升溫溫度的模式及機台噴嘴內的壓力到某特定的參數值,就可以降低染布出現染整顏色不均勻的發生情況。然而\cite{Etemadifar.etc}述說以前的文獻及工廠所做的實驗方法,基本上都是依照其他研究所提出的參數,藉由固定特定參數下更改想要測量的參數加以變動,藉此了解此特定參數是否有對製程有所一定程度的影響,
\cite{Wang.etc}提出供技術人員參考和控制的關鍵因子以達到染色的均勻性,包含pH值、浴比及溫度等因子。藉由這些因子,探究製程參數對染色品質的影響,以期提出關鍵因子,得到穩健性更高的製程參數設置,並預估相關製程參數和染色品質的關係曲面。至於細節,由於第一章範圍與限制當中說明,本研究著重於模型架構,以及其相對應之最佳化模型求解,實驗設計的細節就不再贅述。