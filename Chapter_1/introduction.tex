%!TEX root = ../thesis.tex
\chapter{紡織染色製程背景介紹}
\label{c:intro} %交互參照
近年來新興國家低成本紡織品紛紛興起,大部份的國家採取以量制價的策略,對於台灣的紡織產業已難以與其他勞力密集之國家競爭。為了能與其他新興國家的紡織產業競爭,提高產品本身的價值則為目前台灣紡織業主要的策略,所以製造與行銷機能性紡織品正是台灣與新興國家產品區隔的主要管道。為了提高台灣產品的價值,紡織業者紛紛以高質化的機能性紡織品,配合全方位的行銷服務因應挑戰。由財團法人紡織產業綜合研究所(TTRI)統計出台灣紡織品出口年產值超過3,300億,其中主要以高附加價值之機能布料為主,占全球機能布料出口之7成。台灣具有完整的紡織業產業鏈,包含上游有天然纖維、人造纖維;中上游有紡紗;中游有織布;中下游有染整;下游有成衣業者。

本研究著眼於高單位價值的機能布料染整製程,此製程在機能性紡織品包括布料染色及提供機能性。近年來創造的附加價值逐年提高,在染整製程中,國際品牌服飾對於布料顏色精確度要求越來越高且趨向少量多樣。染整製程中,通常由實驗室產出配方及製程資料,交由染整工廠依配方及製程試染,並設定染程參數,若在驗色階段發現色差、顏色不均、波紋、及固色率不到標準,未達對色率標準之不良品,需再經過漂洗、補色等流程或直接報廢,在染色過程中所產生的不良品影響生產成本甚鉅。因此本研究針對染整過程中,調整製程內的參數因子,
降低染整過程中多餘的資源浪費以及提高對色的品質,進而降低染整過程中所產生的成本。

紡織產業在台灣屬於傳統產業,對於大部分的紡織工廠來說,許多的配方設定以及環境的設定,大多藉由過去的經驗以及微幅調整後的回饋設定染整參數,而主要原因在於染色過程,光是染色劑原料對不同的染布原料就能產生出上萬組情境,如果為了找出因應這大量組合的最佳參數組合,在過程中所需要的成本必定非常可觀,導致目前紡織產業對於染色製程的環境參數的選擇趨近於固定幾種。

本研究想藉由尋找不同的染整過程參數因子,如染劑濃度、助劑濃度、初始pH值、浴比值、主泵浦速率、帶布輪速率、升溫速率、持溫溫度、水洗時間、皂洗升溫速率、皂洗持溫溫度、皂洗時間等\dots,由此控制染整過程中的變異,而達到有效降低資源耗費以及品質,進而節省因染色失敗而造成的巨量成本。

為了達到在不影響品質下,有效降低染整過中不必要的浪費,並能選擇適合的製程參數因子,因此本研究的目的如下:
\begin{enumerate}[(1)]
	\item 建立一個適合紡織業者經常使用的染整製程的最佳化模型框架。
	\item 為了建構模型框架,考量多種環境因子對於本研究的目標特性,找出提供效益最大的模型因子。
	\item 由最佳化求解方法求解環境因子建構的模型,得到高品質以及低成本的組合。
\end{enumerate}
在本研究中,主要分為五個章節,第一章為研究介紹,此章敘述了本研究動機以及目的,以及論文的框架;在第二章的文獻探討,探討目前與本研究主題相關的文獻,以及學界上發展的程度;第三章描述紡織染色製程現況,將詳細介紹目前紡織業者是如何以經驗調整染色過程中的製程參數;第四章為模型建立,綜合文獻及業界的資料,建立對成本以及品質具有影響力的模型,並將估計參數代入模型中;第五章介紹最佳化方法,並且求解數值化的模型;第六章為數值分析,主要分為敏感度分析以及最佳組合與其他組合比較兩個子題討論,期望可藉由數值分析了解參數變動對求解結果之影響為何,且提出對應的管理意涵;第七章為結論與建議,針對建立的模型彙整並提出未來可能行研究方向。

