%!TEX root = ../thesis.tex
\chapter{結論}
\label{c:con}
本研究主要探討染色製程最佳化問題,從大略介紹機能布料對於現今世界的發展性,進而討論藉由降低紡織業在染色過程中的成本,提高紡織產業的利潤;為了有效降製程成本,我們數次拜訪業者,並由化驗室得來的實驗數據進行多項分析後,最後我們使用了五個對於染色過程中具有較大影響力的五個染整製程因子,作為本研究建置模型的決策變數,並根據這些製程參數的特性分別建置運作成本最小化模型、穩定度模型以及品質模型($\Delta E$模型),並藉由序列二次規劃法作為本研究的最佳化方法,分別對三個模型搜尋最佳解,因此分別得到模型的製程參數最佳組合;在數據分析中我們將其他文獻所使用的方法做為比較的對象,以基因演算法針對本研究的三個模型分別求解,得到的製程參數組合做為對照組合,另外我們還從可行區域內隨機抽取其他製程參數組合,最後將所有的製程參數組合與業界常用的組合互相比較。

在數據分析,在我們的假設下得知,序列二次規劃法中的穩定度最佳化模型所求得的製程參數組合,與其他的製程參數組合互相比較下,品質較好且較為穩定且所需要花費的運作成本差異不大,如果以每年250個工作天以及每天需要染整160缸染布下,約可以節省報廢或重染成本約每年三千多萬台幣,對於業者來說是相當可觀的成本花費。

本研究中主要只有探討五個具有顯著的影響成本、品質,以及業者有興趣的製程參數,並且根據運作成本以及品質進行求解以及數值分析,但實際的染整製成參數多達數十種,如在未來有機會能夠再與業者合作,可增加製程參數的種類以及增加實際現場的實驗數據,增加模型的可信度,並可以導入成本導向以及品質導向的目標合併,如多目標方法,在多製程參數下求取成本以及品質之間的最佳平衡點,除了為業者減少報廢成本外還能有效降低其他的成本。
