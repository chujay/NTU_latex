%!TEX root = ../thesis.tex
\begin{abstractzh}
本研究的主要目的為,
在已知染色製程參數以及染色表現因子的關聯下,
建構機能布染色製程參數優化模型,
並考量生產成本以及品質成本,
以達到有效降低成本以及提高品質的穩定程度。
本研究主要分成二個部分,
一部分探討染整製程參數優化技術,
由表現因子關聯分析所估計的製程參數,
建構成本相關的染整製程總成本模型、品質相關的染整製程穩定度模型,
以及與品質標準有直接關係的染整製程對色差異度模型。
另一個部分基於前面部分所建構的模型,
本研究以非線性規劃的Sequential Quadratic Programming (SQP)方法,
分別對三個非凸規劃問題的模型,求解非線性規劃問題,
藉由搜尋方法找出製程運作成本最低、品質穩定度最高
以及對色差異度最小的表現因子組合。
\\ \\關鍵字:非凸集合、非線性規劃、序列二次規劃、穩定度
\end{abstractzh}

\begin{abstracten}
This research is to develop the dyeing parameter optimization model for functional textiles based on the analysis of relationship between the manufacturing parameters in the dyeing process and the dyeing performance. The aim of this research is to minimize the total dyeing cost including the production and energy quality costs with the consideration of robustness measure and dyeing performance.
Based on the relationship the estimated process parameters, the first task is discussion of the dyeing parameter optimization for dyeing process total costs model, dyeing process robustness model with consideration of the quality and the quality model about the chromatic aberration. 
According to the models, we use Sequential Quadratic Programming (SQP) method to solve the non-convex programming problems and search the combination of the maximum robustness of quality and the minimum of process cost and chromatic aberration. 
\\ \\Keywords: non-convex, nonlinear, SQP, robust
\end{abstracten}